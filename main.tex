\documentclass[12pt, a4paper]{article}
\usepackage[italian]{babel}
\usepackage{helvet}
\renewcommand{\familydefault}{\sfdefault}
\usepackage{setspace}
\usepackage[margin=1in]{geometry}
\usepackage{lipsum}
\usepackage{graphicx} % Required for inserting images


\title{Bookkepper test}
\author{Luca Falasca}

\begin{document}

\maketitle
\tableofcontents
\newpage

\section{Introduction}

\section{Journal}
\subsection{listJournalIds}
\subsubsection{Descrizione del metodo}
Lists all journal IDs filtered by a specified journal ID filter.

This method scans the given directory containing journal log files and extracts
journal IDs based on the provided filter. If no filter is provided, all journal
IDs present in the directory are returned.


\paragraph{Input:}
\begin{itemize}
  \item journalDir journal dir : The directory containing journal log files.
  \item filter journal id filter
\end{itemize}
\paragraph{Output:}
    list of filtered ids

\subsubsection{Category Partition}
\paragraph{journalDir}
\begin{itemize}
  \item \{Directory contenente file di log\}
  \item \{Directory contenente file di log e altri file\}
  \item \{Directory contenente file non di log\}
  \item \{path non esistente\}
  \item \{Path di un file\}
  \item null
\end{itemize}
Siccome la variabile journalDir è il path di una directory, ho partizionato 
il dominio in base al contenuto della directory e al suo effettivo utilizzo, rendendo la scelta delle partizioni 
una conseguenza del fatto che il metodo lavora su file di log.

\paragraph{JournalIdFilter}
\begin{itemize}
    \item Filtro esistente
    \item Filtro sempre True
    \item Filtro sempre False
    \item Filtro inesistente
    \item null
\end{itemize}

\subsubsection{Boundary Analysis}
Andiamo a definire per ogni partizione i Boundary values

\paragraph{journalDir}
\begin{itemize}
  \item \{Directory contenente 1 file di log\}
  \item \{Directory contenente 1 file di log e 1 file di testo\}
  \item \{Directory contenente 1 file di testo\}
  \item \{path non esistente\}
  \item \{Path di un file di log\}
  \item null
\end{itemize}

\paragraph{JournalIdFilter}
\begin{itemize}
    \item JournalRollingFilter  -- Questo filtro è l'unico filtro esistente utilizzato nell'applicazione, 
    tuttavia siccome viene utilizzato in un contesto molto specifico dell'applicazione e non è un filtro generico, l'ho rimpiazzato
    con una sua versione semplificata più generale. Altrimenti non sarebbe stato adatto ad un test di unità, ma sarebbe stato 
    più un test di integrazione
    \item Filtro sempre True
    \item Filtro sempre False
    \item MyFilter -- Questo filtro è un filtro personalizzato creato appositamente per questo test che va a filtrare i journal
    in base al loro nome, in particolare se il journalId è > 0 viene accettato, altrimenti no. 
    Questo filtro ha lo scopo di testare come si comporta il sistema in caso di definizione di un nuovo filtro non già esistente 
    nel sistema e quindi non è importante il tipo di filtraggio che fa.
    \item null
\end{itemize}

Siccome i due parametri di input sono abbastanza scorrelati tra loro ha più senso adottare un approccio 
unidimensionale piuttosto che uno multidimensionale che sarebbe più adatto quando ci sono delle interazioni forti
e che portano alla necessità di testare tutte le combinazioni tra i parametri. Inoltre avendo un approccio multidimensionale
si finirebbe probabilemente per avere molti test non rilevanti e vanno a coprire scenari già coperti, e quindi sarebbero inutili.
ù
Enumeriamo ora i casi di test

\begin{table}[ht]
\centering
\begin{tabular}{|c|c|c|}
\hline
journalDir & JournalIdFilter & Risultato Atteso \\
\hline
{Directory contenente 1 file di log} & JournalRollingFilter & ... \\
{Directory contenente 1 file di log} & Filtro sempre True & ... \\
{Directory contenente 1 file di log} & Filtro sempre False & ... \\
{Directory contenente 1 file di log} & MyFilter & ... \\
{Directory contenente 1 file di log} & null & Exception \\
{Directory contenente 1 file di log e 1 file di testo} & JournalRollingFilter & ... \\
{Directory contenente 1 file di log e 1 file di testo} & Filtro sempre True & ... \\
{Directory contenente 1 file di log e 1 file di testo} & Filtro sempre False & ... \\
{Directory contenente 1 file di log e 1 file di testo} & MyFilter & ... \\
{Directory contenente 1 file di log e 1 file di testo} & null & Exception \\
{Directory contenente 1 file di testo} & JournalRollingFilter & ... \\
{Directory contenente 1 file di testo} & Filtro sempre True & ... \\
{Directory contenente 1 file di testo} & Filtro sempre False & ... \\
{Directory contenente 1 file di testo} & MyFilter & ... \\
{Directory contenente 1 file di testo} & null & Exception \\
{path non esistente} & JournalRollingFilter & [ ] \\
{path non esistente} & Filtro sempre True & [ ] \\
{path non esistente} & Filtro sempre False & [ ] \\
{path non esistente} & MyFilter & [ ] \\
{path non esistente} & null & [ ] \\
{Path di un file di log} & JournalRollingFilter & ... \\
{Path di un file di log} & Filtro sempre True & ... \\
{Path di un file di log} & Filtro sempre False & ... \\
{Path di un file di log} & MyFilter & ... \\
{Path di un file di log} & null & Exception \\
null & JournalRollingFilter & Exception \\
null & Filtro sempre True & Exception \\
null & Filtro sempre False & Exception \\
null & MyFilter & Exception \\
null & null & Exception \\
\hline
\end{tabular}
\end{table}

\subsubsection{Adequacy Control Flow}
Ora per verificare l'adeguatezza dei casi di test, vado a definire dei critedi di adeguatezza

Criteri di adeguatezza black box:
documentazione a disposizione:

javadoc
/**
* List all journal ids by a specified journal id filer.
*
* @param journalDir journal dir
* @param filter journal id filter
* @return list of filtered ids
*/

 documentazione apache:
 Journals
A journal file contains BookKeeper transaction logs. 
Before any update to a ledger takes place, the bookie ensures that a transaction describing 
the update is written to non-volatile storage. A new journal file is created once the bookie 
starts or the older journal file reaches the journal file size threshold.

Possiamo dedurre da queste documentazioni che questo metodo ricava gli id dei journal in una 
determinata cartella, filtrandoli in base ad un filtro specificato, se non viene specificato 
nessun filtro, allora vengono restituiti tutti gli id dei journal presenti nella cartella.

La Figura \ref{fig:listJournalIdsDiagram} mostra un diagrammma 
funzionale del metodo listJournalIds

A partire da questo definisco i seguenti criteri di adeguatezza:
- almeno un test con un filtro
- almeno un test senza filtro
- almeno un test senza journal nella cartella
- almeno un test con almeno un journal nella cartella

In questo caso i criteri di adeguatezza sono già stati soddisfatti 
dai casi di test precedentemente descritti


Dal diagramma precedente (Figura \ref{fig:listJournalIdsDiagram}) possiamo 
ricavare un control flow graph (Figura \ref{fig:listJournalIdsCFG}), per poi
utilizzare dei criteri di copertura basati sul control flow

Dato che ho un approccio black box e quindi non sto utilizzando il codice sorgente per verificare l'adeguatezza dei casi di test,
ipotizzo in base agli input se un caso di test andrà a coprire un certo arco del CFG o meno.

Dato che in questa fase sto avendo un approccio black box, e quindi basato sulla funzionalità del metodo, 
eviterò di utilizzare la coverage come parametro di adeguatezza, in quanto non è possibile ricavare
dall'output del metodo se un certo arco è stato coperto o meno, ma solo se il risultato è corretto oppure no.
Quindi valuterò la copertura del CFG solo in una fase successiva, quando prenderò in considerazione
anche il codice sorgente (approccio whitebox), utilizzando la coverage come parametro di adeguatezza.

Quello che farò in questa fase è quindi valutare se i casi di test che ho definito in precedenza
coprono tutti gli archi del CFG, in base alla mia conoscenza del metodo e dei suoi input, e nel caso
aggiungere quelli mancanti.


Arco (1,2)  -- textgreater Tutti i test che hanno almeno un file di log coprono questo caso, quindi non è necessario aggiungere ulteriori test
Arco (2,3)  - \> Tutti i test che hanno almeno un file di log e un filtro non null coprono questo caso, quindi non è necessario aggiungere ulteriori test
Arco (2,4)  -\> 
Questo arco non è coperto perchè manca un test che abbia almeno un file di log e il filtro null. Basta quindi aggiungere il 
test necessario:
\begin{table}[ht]
  \centering
  \begin{tabular}{|c|c|c|}
  \hline
    {Directory contenente 1 file di log} & null & Exception \\
  \end{tabular}
\end{table}

Arco (3,4)  -\> Tutti i test che hanno almeno un file di log e un filtro non null coprono questo caso, quindi non è necessario aggiungere ulteriori test



Non è stato necessario aggiungere nè classi di equivalenza nè casi di test,
in quanto quelli già definiti soddisfano i criteri di adeguatezza precedentemente definiti

Ora andiamo a valutare la coverage ottenuta con jacoco con i casi di test definiti
fino ad ora (Figura \ref{fig:JacocoCoveragelistJournalIds1.png})

Come si può vedere dalla figura \ref{fig:JacocoCoveragelistJournalIds1.png},
la statement coverage è del 100\% (colonna missed instruction), 
e quindi tutte le linee di codice del metodo sono state eseguite

Invece la condition coverage (colonna missed branches) e del 91\%, ed è quindi migliorabile.
Andando a vedere in dettaglio il report, si può notare che la condizione che non è stata coperta
è una delle 4 combinazioni dell'if a riga 106 come si vede dalla figura \ref{fig:MissedBranchListJournalIds.png}

La condizione presa in esame è la seguente:
logFiles == null || logFiles.length == 0

il branch mancato è quello dove logFiles == null, questo perchè se il parametro di input journalDir è null non esegue proprio il metodo listFiles()


questo è dovuto al fatto che il path non era esistente e quindi il metodo tornava null

Andiamo quindi a definire un caso di test che copra questo branch
Per farlo andiamo semplicemente basta aggiungere un category partition che copre il caso di una cartella esistente ma vuota

\begin{table}[ht]
\centering
\begin{tabular}{|c|c|c|}
  \hline
  journalDir & JournalIdFilter & Risultato Atteso \\
  \hline
  {Directory vuota} & null & [ ] \\
  \hline
\end{tabular}
\end{table}

Ora quindi l'insieme dei test è il seguente

\begin{table}[ht]
  \centering
  \begin{tabular}{|c|c|c|}
  \hline
  journalDir & JournalIdFilter & Risultato Atteso \\
  \hline
  {Directory contenente 1 file di log} & JournalRollingFilter & ... \\
  {Directory contenente 1 file di log} & Filtro sempre True & ... \\
  {Directory contenente 1 file di log} & Filtro sempre False & ... \\
  {Directory contenente 1 file di log} & MyFilter & ... \\
  {Directory contenente 1 file di log} & null & Exception \\
  {Directory contenente 1 file di log e 1 file di testo} & JournalRollingFilter & ... \\
  {Directory contenente 1 file di log e 1 file di testo} & Filtro sempre True & ... \\
  {Directory contenente 1 file di log e 1 file di testo} & Filtro sempre False & ... \\
  {Directory contenente 1 file di log e 1 file di testo} & MyFilter & ... \\
  {Directory contenente 1 file di log e 1 file di testo} & null & Exception \\
  {Directory contenente 1 file di testo} & JournalRollingFilter & ... \\
  {Directory contenente 1 file di testo} & Filtro sempre True & ... \\
  {Directory contenente 1 file di testo} & Filtro sempre False & ... \\
  {Directory contenente 1 file di testo} & MyFilter & ... \\
  {Directory contenente 1 file di testo} & null & Exception \\
  {path non esistente} & JournalRollingFilter & [ ] \\
  {path non esistente} & Filtro sempre True & [ ] \\
  {path non esistente} & Filtro sempre False & [ ] \\
  {path non esistente} & MyFilter & [ ] \\
  {path non esistente} & null & [ ] \\
  {Path di un file di log} & JournalRollingFilter & ... \\
  {Path di un file di log} & Filtro sempre True & ... \\
  {Path di un file di log} & Filtro sempre False & ... \\
  {Path di un file di log} & MyFilter & ... \\
  {Path di un file di log} & null & Exception \\
  null & JournalRollingFilter & Exception \\
  null & Filtro sempre True & Exception \\
  null & Filtro sempre False & Exception \\
  null & MyFilter & Exception \\
  null & null & Exception \\
  {Directory vuota} & null & [ ] \\
  \hline
  \end{tabular}
  \end{table}

e grazie a questo test anche la condition coverage è del 100\% (Figura \ref{fig:JacocoCoveragelistJournalIds2.png}),
 con un aumento del 9%.

 \subsubsection{Adequacy Control Flow}

 Andiamo ora ad fare un lavoro di adeguatezza del dataflow utilizzando il framework badua per calcolare la all-uses coverage

 \subsubsection{Mutation Testing}




















\subsection{logAddEntry}
\subsubsection{Descrizione del metodo}
record an add entry operation in journal.
\paragraph{Input:}
\begin{itemize}
  \item long ledgerId
  \item long entryId
  \item ByteBuf entry
  \item boolean ackBeforeSync
  \item WriteCallback cb
  \item Object ctx
  \item journalStats
  \item memoryLimitController
  \item queue
  \item callbackTime
\end{itemize}
\paragraph{Output:}
niente

\subsubsection{Category Partition}

\paragraph{journalId}
\begin{itemize}
  \item 
\end{itemize}



\subsubsection{Boundary Analysis}
Andiamo a definire per ogni partizione i Boundary values

\subsubsection{Test Suit}
Enumeriamo ora i casi di test








\subsection{scanJournal}
\subsubsection{Descrizione del metodo}
 * Scan the journal.
 *
 * @param journalId Journal Log Id
 * @param journalPos Offset to start scanning
 * @param scanner Scanner to handle entries
 * @return scanOffset - represents the byte till which journal was read
 * @throws IOException
\paragraph{Input:}
\begin{itemize}
  \item journalId
  \item journalPos
  \item journalDirecotry
  \item journalPreAllocSize
  \item journalWriteBufferSize
  \item conf
  \item fileChannelProvider
\end{itemize}
\paragraph{Output:}
Il byte fino a dove è stato letto il journal

\subsubsection{Category Partition}

\paragraph{journalId}
\begin{itemize}
  \item 
\end{itemize}



\subsubsection{Boundary Analysis}
Andiamo a definire per ogni partizione i Boundary values

\subsubsection{Test Suit}
Enumeriamo ora i casi di test



\section{Resources}
\subsection{Tables}
\subsection{Images}
  \begin{figure}
    \includegraphics[width=\linewidth]{listJournalIds.jpg}
    \caption{Funcitonal diagram di listJournalId}
    \label{fig:listJournalIdsDiagram}
  \end{figure}
  \begin{figure}
    \includegraphics[width=\linewidth]{CFGlistJournalIds.jpg}
    \caption{Control Flow Graph di listJournalIds}
    \label{fig:listJournalIdsCFG}
  \end{figure}

  \begin{figure}
    \includegraphics[width=\linewidth]{JacocoCoveragelistJournalIds1.png}
    \caption{Jacoco coverage dilistJournalIds}
    \label{fig:listJournalIdsCFG}
  \end{figure}

  \begin{figure}
    \includegraphics[width=\linewidth]{MissedBranchListJournalIds.png}
    \caption{Jacoco coverage dilistJournalIds}
    \label{fig:listJournalIdsCFG}
  \end{figure}

  \begin{figure}
    \includegraphics[width=\linewidth]{JacocoCoveragelistJournalIds2.png}
    \caption{Jacoco coverage dilistJournalIds}
    \label{fig:listJournalIdsCFG}

  \end{figure}

\end{document}
